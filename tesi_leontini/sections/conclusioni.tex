\clearpage
\thispagestyle{empty} % Ensures the current blank page has no header/footer
\cleardoublepage % Ensures the next page (where the chapter starts) is a right-hand page

\chapter{Conclusioni}
\label{chap:conclusioni}

Di seguito sono descritte le sezioni che, in generale, dovrebbero comporre una tesi di laurea. La suddivisione presentata non è strettamente vincolante e dovrebbe essere adattata in base alle esigenze e alle caratteristiche della tesi trattata.

\section{Conclusioni}

L’obiettivo di questa tesi è stato quello di progettare e sviluppare un’applicazione web performante per la visualizzazione interattiva, il confronto e l’esportazione di serie temporali di dati, in particolare in ambito scientifico e ingegneristico. La soluzione proposta si fonda su un’architettura moderna che integra Astro per il rendering server-side, SolidJS per la gestione reattiva dell’interfaccia utente e D3.js per la visualizzazione dinamica dei dati, con MySQL come backend per la gestione efficiente di serie temporali anche di grandi dimensioni.

L’applicazione realizzata consente di esplorare e confrontare diversi scenari di dati, navigare in modo storico tramite la gestione avanzata dello stato via URL, ed esportare facilmente le visualizzazioni per analisi e reporting offline. L’adozione di un’architettura modulare e di strategie di ottimizzazione mirate ha permesso di raggiungere elevati livelli di performance e reattività, anche su dispositivi mobili o in condizioni di connettività limitata.

I risultati ottenuti mostrano come la soluzione sia in grado di gestire dataset di grandi dimensioni mantenendo tempi di risposta ridotti e fluidità nelle interazioni, confermando la validità delle scelte tecnologiche e architetturali adottate. Le funzionalità di confronto tra stati, esportazione e navigazione storica sono risultate particolarmente apprezzate dagli utenti durante le fasi di test e validazione.

Nonostante i risultati positivi, la soluzione proposta presenta alcune limitazioni. In particolare, la gestione di dataset estremamente voluminosi potrebbe richiedere ulteriori ottimizzazioni lato backend, come l’adozione di tecniche di indicizzazione avanzata, caching o l’utilizzo di database NoSQL per particolari casi d’uso. Inoltre, l’integrazione con fonti dati eterogenee o in tempo reale rappresenta una possibile evoluzione futura, così come lo sviluppo di ulteriori componenti di analisi avanzata e reporting automatico.

Il codice sorgente dell’applicazione è disponibile pubblicamente su GitHub all’indirizzo: \url{https://github.com/aleleo1/project}

Eventuali risultati di ricerca o pubblicazioni derivanti dal presente lavoro saranno menzionati e resi disponibili nella piattaforma open access dell’ateneo.

In conclusione, questa tesi ha fornito un contributo concreto al tema della visualizzazione performante di dati scientifici in ambiente web, proponendo una soluzione moderna, efficiente e facilmente estendibile, che potrà fungere da base per sviluppi futuri nel campo della data visualization e dell’analisi interattiva dei dati.