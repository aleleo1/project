\clearpage{\pagestyle{empty}\cleardoublepage}
\chapter{Stato dell'arte}
\label{chap:stato_arte}

Di seguito sono descritte le sezioni che, in generale, dovrebbero comporre una tesi di laurea. La suddivisione presentata non è strettamente vincolante e dovrebbe essere adattata in base alle esigenze e alle caratteristiche della tesi trattata.

\section{Stato dell'arte}

Nel contesto della visualizzazione e analisi di serie temporali di dati scientifici e ingegneristici, esistono numerose soluzioni e piattaforme che affrontano, con diversi approcci, le criticità legate all’efficienza, alla scalabilità e all’interattività. In questo capitolo vengono analizzate le principali soluzioni esistenti, valutandone le caratteristiche, i punti di forza e le limitazioni, con particolare attenzione alle differenze rispetto alla soluzione proposta in questa tesi.

\subsection{Grafana}

Grafana è una delle piattaforme open source più diffuse per la visualizzazione e il monitoraggio di serie temporali di dati. Offre un’interfaccia utente altamente personalizzabile, il supporto a numerosi database (tra cui Prometheus, InfluxDB, MySQL) e la possibilità di creare dashboard interattive. Grafana è particolarmente apprezzata per la sua modularità e per l’ampia disponibilità di plugin.

\textbf{Vantaggi:}
\begin{itemize}
    \item Interfaccia user-friendly e altamente personalizzabile.
    \item Ampio supporto a fonti dati eterogenee.
    \item Plugin per integrazione con sistemi di alerting e notifiche.
    \item Attiva comunità open source.
\end{itemize}

\textbf{Svantaggi:}
\begin{itemize}
    \item Alcune funzionalità avanzate richiedono la versione enterprise.
    \item La gestione di dataset molto grandi può risultare onerosa in termini di risorse.
    \item Lato client non sempre ottimizzato per dispositivi mobili.
    \item La condivisione dello stato tramite URL non è sempre completa e trasparente.
\end{itemize}

\textbf{Differenze rispetto alla soluzione proposta:}
La soluzione proposta, rispetto a Grafana, si concentra su una gestione più efficiente dello stato tramite URL, sull’esportazione avanzata delle visualizzazioni e su una maggiore ottimizzazione delle prestazioni lato client, soprattutto in ambiente mobile.

\subsection{Kibana}

Kibana è un tool open source principalmente utilizzato per la visualizzazione di dati provenienti da Elasticsearch, ma offre anche funzionalità avanzate di esplorazione e analisi di dati temporali. Kibana consente la creazione di dashboard e report personalizzati, con un focus particolare su dati di log e metriche di sistema.

\textbf{Vantaggi:}
\begin{itemize}
    \item Integrazione nativa con Elastic Stack.
    \item Potenti funzionalità di ricerca, filtro e aggregazione.
    \item Dashboard interattive e personalizzabili.
\end{itemize}

\textbf{Svantaggi:}
\begin{itemize}
    \item Dipendenza da Elasticsearch come backend dati.
    \item Scalabilità e performance dipendono dalla configurazione del cluster.
    \item Curva di apprendimento relativamente elevata per utenti non tecnici.
\end{itemize}

\textbf{Differenze rispetto alla soluzione proposta:}
La soluzione proposta offre maggiore flessibilità nella scelta della sorgente dati e una più semplice esportazione dei risultati, oltre a una maggiore leggerezza e semplicità di installazione.

\subsection{Plotly Dash}

Plotly Dash è un framework Python per la creazione di applicazioni web interattive di visualizzazione dati. Consente la realizzazione di dashboard dinamiche, con grafici di elevata qualità e interattività avanzata.

\textbf{Vantaggi:}
\begin{itemize}
    \item Facilità di utilizzo per chi ha familiarità con Python.
    \item Ampia gamma di visualizzazioni e grafici interattivi.
    \item Possibilità di integrare facilmente altri moduli Python per analisi dati.
\end{itemize}

\textbf{Svantaggi:}
\begin{itemize}
    \item Performance limitate su dataset di grandi dimensioni.
    \item Scalabilità server-side non sempre ottimale senza configurazioni avanzate.
    \item Lato client non sempre reattivo per interazioni complesse.
\end{itemize}

\textbf{Differenze rispetto alla soluzione proposta:}
La soluzione proposta, essendo sviluppata interamente in ambiente JavaScript/TypeScript e con tecnologie moderne come SolidJS e Astro, garantisce migliore reattività lato client e un’integrazione più immediata con ecosistemi web già esistenti.

\subsection{Highcharts Cloud}

Highcharts Cloud è una piattaforma SaaS che permette di creare, personalizzare e condividere grafici interattivi direttamente dal browser.

\textbf{Vantaggi:}
\begin{itemize}
    \item Interfaccia intuitiva e immediata.
    \item Ampia varietà di grafici disponibili.
    \item Possibilità di esportazione in diversi formati.
\end{itemize}

\textbf{Svantaggi:}
\begin{itemize}
    \item Funzionalità avanzate disponibili solo a pagamento.
    \item Limitazioni nella personalizzazione dei flussi dati e delle interazioni.
    \item Non adatto a integrazioni personalizzate o workflow complessi.
\end{itemize}

\textbf{Differenze rispetto alla soluzione proposta:}
La piattaforma sviluppata in questa tesi consente una piena integrazione con flussi dati personalizzati e la gestione avanzata dello stato tramite URL, offrendo maggiore flessibilità e possibilità di estensione.

\subsection{Altre soluzioni open source e librerie JavaScript}

Esistono numerose librerie JavaScript per la visualizzazione di serie temporali, tra cui D3.js, Chart.js, e ECharts. Queste librerie forniscono strumenti potenti per la creazione di grafici, ma richiedono spesso sviluppi ad hoc per integrare funzionalità avanzate come la gestione dello stato, il confronto tra scenari, o l’esportazione delle visualizzazioni.

\textbf{Vantaggi:}
\begin{itemize}
    \item Estrema flessibilità e possibilità di personalizzazione.
    \item Ampia comunità di supporto.
    \item Elevate performance se utilizzate correttamente.
\end{itemize}

\textbf{Svantaggi:}
\begin{itemize}
    \item Necessità di competenze avanzate di sviluppo front-end.
    \item Mancanza di funzionalità pronte all’uso per la gestione avanzata dello stato o l’esportazione.
    \item Maggiore effort di integrazione con sistemi backend.
\end{itemize}

\textbf{Differenze rispetto alla soluzione proposta:}
La soluzione proposta integra nativamente funzionalità di gestione avanzata dello stato, esportazione e confronto, riducendo così il tempo e l’effort richiesti per ottenere un prodotto finale pronto all’uso.

\section{Sintesi e confronto}

Dall’analisi delle soluzioni esistenti emerge che, sebbene siano disponibili strumenti potenti e flessibili per la visualizzazione di dati temporali, nessuna delle piattaforme considerate offre contemporaneamente:
\begin{itemize}
    \item prestazioni elevate anche su dataset di grandi dimensioni;
    \item gestione avanzata dello stato e della navigazione tramite URL;
    \item funzionalità di confronto tra stati multipli;
    \item esportazione facile e personalizzata delle visualizzazioni;
    \item massima compatibilità e performance su dispositivi mobili e desktop.
\end{itemize}

La soluzione proposta in questa tesi si colloca quindi come un’alternativa innovativa, in grado di coniugare le migliori pratiche di sviluppo web moderno con funzionalità avanzate e performance ottimizzate per le esigenze della ricerca scientifica e dell’industria.
