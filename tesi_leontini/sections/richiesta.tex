\clearpage{\pagestyle{empty}\cleardoublepage}
\chapter{Modalità di richiesta e svolgimento della tesi}
\label{chap:tesi}

Il presente documento contiene note specifiche per la richiesta e stesura di tesi in Databases and Web Programming (\textbf{DBWP}) e in Cognitive Computing and Artificial Intelligence (\textbf{CCAI}).

\section{Adempimenti burocratici}

La conoscenza e il rispetto delle tempistiche relative agli adempimenti burocratici (ad es., richiesta di tesi, domanda di laurea, caricamento della tesi, caricamento della presentazione) sono responsabilità dello studente.

Lo studente è tenuto inoltre ad assicurarsi che tali adempimenti vengano espletati \emph{con sufficiente anticipo} rispetto alla scadenza. In particolare, sarà cura dello studente tenere conto delle tempistiche relative \emph{alla correzione} da parte del docente degli elaborati prodotti (tesi, presentazione, ecc.).

\section{Richiesta della tesi}

\textbf{DBWP} --- La richiesta della tesi prevede che lo studente \textbf{proponga} un'applicazione da realizzare, \textbf{previa analisi} delle soluzioni esistenti che affrontano la tematica scelta e descrizione degli aspetti per cui l'applicazione proposta rappresenta un miglioramento o una variante degna di interesse. In altre parole, al momento della proposta dell'applicazione non è sufficiente dire che tipo di applicazione si vorrebbe realizzare, \ul{ma è necessario anche elencare le soluzioni già esistenti e spiegare in che modo la proposta si differenzia da queste}. 

\noindent\textbf{CCAI} --- Lo studente è incoraggiato a proporre un'idea in maniera analoga a quanto descritto sopra; in alternativa, verranno suggerite delle tematiche in attinenza con l'attività di ricerca del docente.

\section{Definizione e svolgimento dell'attività di tesi}

Qualora la proposta di tesi venga condivisa e accettata, verranno stabiliti gli obiettivi minimi il cui raggiungimento rappresenterà la condizione necessaria affinché il lavoro di tesi possa considerarsi completo. 

Il docente guiderà e supporterà lo studente nelle scelte architetturali e nelle difficoltà metodologiche; la soluzione di problemi legati ad aspetti implementativi è responsabilità dello studente. Durante lo svolgimento della tesi, saranno effettuati degli incontri periodici (ogni 1 o 2 settimane) al fine di monitorare lo stato dei lavori.

\noindent\textbf{DBWP} --- L'applicazione proposta deve essere sviluppata integrando tecnologie studiate nell'ambito del corso di Databases and Web Programming e \textbf{tecnologie studiate indipendentemente dallo studente come parte dell'attività di tesi}, concordate col docente (ad esempio, Node.js, React, AngularJS, AWS).

\section{Scrittura della tesi}

La tesi deve essere redatta in \LaTeX, un linguaggio di markup (lo stesso usato per la stesura di questo documento, il cui codice sorgente verrà fornito come template da cui iniziare la stesura). La stesura sarà effettuata tramite la piattaforma web OverLeaf, che integra un compilatore \LaTeX~e permette di condividere il documento col docente. A questo link è disponibile una breve introduzione a \LaTeX: \url{https://it.overleaf.com/learn/latex/Learn_LaTeX_in_30_minutes}

Lo studente può scegliere se scrivere la tesi in italiano o in inglese.

La correzione della tesi avviene per fasi, attraverso cicli di scrittura e revisione dei singoli capitoli, nell'ordine con cui questi appariranno nel manoscritto finale. Al completamento della tesi, sarà effettuata un'ulteriore fase di revisione finale. Ciascuna iterazione di revisione richiede fino a un massimo di 3 giorni; per un dato capitolo, in genere sono necessari 2-3 cicli di revisione (ovviamente, durante la revisione di un capitolo lo studente può proseguire con la stesura dei successivi). Gli studenti sono invitati a tenere conto di queste tempistiche in modo da rispettare la scadenze previste. In generale, \textbf{il processo di correzione e revisione della tesi deve essere avviato almeno 20 giorni prima della scadenza per la consegna.}

\section{Valutazione della tesi}

\subsection{Metodologia}

La valutazione della tesi concerne principalmente la correttezza delle metodologie applicate alla risoluzione del problema in oggetto. Per le tesi di natura esplorativa, il raggiungimento di risultati migliori dello stato dell'arte non è un requisito necessario affinché la tesi sia valutata positivamente, a condizione che lo studente sia in grado di dimostrare, sperimentalmente, che sono stati fatti tutti i tentativi ragionevolmente possibili (in termini di tempi richiesti e di complessità) per raggiungere l'obiettivo inizialmente fissato.

In presenza di gravi errori metodologici, la tesi non sarà considerata sufficiente. In presenza di errori metodologici minori o qualora il docente ritenga che non siano stati fatti tutti gli sforzi ragionevolmente possibili, la tesi sarà considerata sufficiente ma la valutazione non sarà massima.

\subsection{Ortografia e sintassi}

L'attività di correzione della tesi riguarda principalmente gli aspetti tecnici e metodologici del problema in oggetto e della soluzione proposta. In presenza di errori puramente linguistici, il docente li evidenzierà, lasciando allo studente il compito di correggerli. In presenza di errori sistematici non corretti, questi influenzeranno la valutazione della tesi.

\section{Tempistiche stimate}

Alla luce dei piani di studio (aggiornati all'a.a. 2021/2022), supponendo un impegno \textbf{a tempo pieno} e tenendo conto dei tempi di attesa (ad es., legati alla comunicazione col docente e alla correzione), le tempistiche \textbf{stimate} per il completamento del lavoro di tesi (inclusa la scrittura della tesi stessa) sono le seguenti:
\begin{itemize}
    \item \textbf{DBWP} --- Tre mesi
    \item \textbf{CCAI} --- Quattro mesi
\end{itemize}

Tali tempistiche sono fornite solo a titolo indicativo, in modo da facilitare la pianificazione delle attività (in particolare, in modo da fornire un'indicazione sulla sessione di laurea alla quale lo studente potrà ragionevolmente puntare). I tempi effettivi dipendono comunque dal raggiungimento degli obiettivi stabiliti in fase di definizione del progetto di tesi.