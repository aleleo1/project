\clearpage{\pagestyle{empty}\cleardoublepage}
\chapter{Struttura della tesi}
\label{chap:struttura}

Di seguito sono descritte le sezioni che, in generale, dovrebbero comporre una tesi di laurea. La suddivisione presentata non è strettamente vincolante e dovrebbe essere adattata in base alle esigenze e alle caratteristiche della tesi trattata.

\section{Sommario}

Lunghezza: una pagina.

Riassumere l'obiettivo della tesi, le motivazioni alla base del problema affrontato e della solzuione proposta, le tecnologie/metodologie utilizzate, i risultati ottenuti.

Il sommario deve essere anche tradotto in inglese.

\section{Introduzione}

Lunghezza: 5-10 pagine.

Presentare una narrativa che giustifichi e introduca il lavoro svolto, attraverso l'elaborazione dei seguenti elementi:
\begin{itemize}
 \item contesto in cui si colloca la tesi;
 \item problema che la tesi si propone di affrontare;
 \item motivi per cui il problema affrontato è importante;
 \item motivi per cui la risoluzione del problema non è banale;
 \item motivi per cui le soluzioni esistenti non sono sufficienti;
 \item obiettivi della tesi;
 \item presentazione, a grandi linee, della soluzione proposta e dei suoi vantaggi;
 \item introduzione alle tecnologie/metodologie utilizzate;
 \item breve cenno ai risultati ottenuti.
\end{itemize}

\section{Stato dell'arte}

Lunghezza: 10-15 pagine.

Descrivere le soluzioni e gli approcci attualmente esistenti per il problema che si sta affrontando.

\textbf{DBWP} --- Fare riferimento ai siti web attualmente esistenti, riportandone caratteristiche, vantaggi, svantaggi e differenze con la soluzione che si propone. Idealmente, ogni soluzione esistente dovrebbe corrispondere ad una sezione del capitolo.

\textbf{CCAI} --- Fare riferimento agli articoli scientifici che hanno affrontato la tematica in questione, descrivendoli brevemente (4-5 righe ciascuno) e confrontandoli con la soluzione che si propone.

La Sez.~\ref{sec:bibliografia} spiega come strutturare i riferimenti bibliografici.

\section{Tecnologie (solo DBWP)}

Lunghezza: 15-20 pagine.

Descrivere le tecnologie utilizzate nella realizzazione del proprio progetto.
t
\section{Soluzione proposta (solo DBWP)}

Lunghezza: 20-30 pagine.

Descrivere l'architettura della soluzione proposta, tramite diagrammi ER e di interazione tra le pagine. Descrivere singolarmente il funzionamento di ciascuna pagina, facendo anche riferimento alle caratteristiche in termini di sicurezza e di usabilità. Inserire opportunamente screenshot dell'applicazione realizzata. 


\section{Conclusioni}

Lunghezza: 1-3 pagine.

Riassumere l'obiettivo della tesi, la soluzione proposta e i risultati ottenuti. Discutere le limitazioni della soluzione proposta e in che modo queste potrebbero essere superate. Qualora il codice sia disponibile online (ad esempio, su GitHub), specificare l'indirizzo. Se il lavoro è stato oggetto di una pubblicazione scientifica, menzionarla.

\section{Bibliografia}
\label{sec:bibliografia}

Inserire solo riferimenti da fonti ``ufficiali'': articoli scientifici o pagine web ufficiali degli autori dei lavori o delle tecnologie che sono state utilizzate. \textbf{Non} utilizzare fonti quali Wikipedia, StackOverflow, Medium, blog, ecc.

La bibliografia deve essere realizzata e gestita tramite BibTeX (questo template carica automaticamente i riferimenti bibliografici dal file \texttt{biblio.bib}). Una guida a BibTeX è disponibile a questo link: \url{https://it.overleaf.com/learn/latex/Bibliography_management_with_bibtex}. La bibliografia alla fine di questo documento presenta un esempio \cite{esempio_citazione}.