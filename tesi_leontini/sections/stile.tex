\clearpage{\pagestyle{empty}\cleardoublepage}
\chapter{Linee guida sullo stile}
\label{chap:style}

\section{Citazioni}

I riferimenti alle voci bibliografiche, inseriti tramite \verb|\cite| (ad es., \cite{esempio_citazione}), sono necessari per giustificare un'affermazione inserita nella tesi. Ad esempio:

\noindent\emph{Già nel 1980 Michael McCloskey e Neal J. Cohen evidenziarono che le architetture connessioniste non sono adatte ad apprendere in maniera incrementale e continua~\cite{mccloskey1989catastrophic}.}

Un altro uso dei riferimenti bibliografici permette di specificare le risorse di informazioni relative all'oggetto del discorso. Ad esempio:

\noindent\emph{Per la realizzazione dell'applicazione lato server è stato utilizzato PHP~\cite{php}.}

Le citazioni vanno inserite prima della punteggiatura di chiusura di una frase, lasciando uno spazio dalla parola che precede. Idealmente, lo spazio dovrebbe essere un \emph{non-breaking space}, ovvero il carattere \texttt{\~}, per evitare che la citazione vada su una nuova riga: ad es., \verb|PHP~\cite{php}|. Guardate la bibliografia alla fine di questo documento per vedere come vengono esplicitati i riferimenti nei precedenti esempi. 
Si noti che il termine ``citazione'' non richiede che venga testualmente citato un brano o un frammento di testo di una fonte esterna: potete esprimere il concetto a parole vostre, per poi aggiungere la citazione a supporto dell'affermazione.

\section{Utilizzo del \emph{corsivo}}

Il corsivo, realizzato tramite il comando \verb|\emph| (\textbf{non} usare \verb|\textit| per il corsivo), è da impiegare nei seguenti casi:
\begin{itemize}
 \item enfasi;
 \item termini in inglese (solo la prima volta che vengono utilizzati).
\end{itemize}

\section{Utilizzo di caratteri \texttt{monospace}}

L'uso del comando \verb|\texttt| per la scrittura di testo in caratteri monospace è da impiegare nei seguenti casi:
\begin{itemize}
 \item nomi di variabili, classi, funzioni, ecc. in linguaggio di programmazione;
 \item nomi di file.
\end{itemize}

\section{Abbreviazioni}

Tutte le abbreviazioni devono essere definite prima di essere utilizzate, presentando al lettore il termine completo con, tra parentesi, l'abbreviazione.

\section{Virgolette}

Le ``virgolette'' si aprono con \verb|``| (doppio backtick; guardate il sorgente del presente documento per capire meglio che carattere sia) e si chiudono con \verb|''| (due apici singoli).

\section{Tabelle}

\`E consigliato l'uso del package \texttt{booktabs} per la realizzazione di tabelle. La documentazione è disponibile \href{https://ctan.mirror.garr.it/mirrors/ctan/macros/latex/contrib/booktabs/booktabs.pdf}{a questo link}.

Un esempio è riportato in Tab.~\ref{tab:esempio}.

\begin{table}
\centering
\begin{tabular}{lccc}
\toprule
\textbf{Model} & \textbf{Precision} & \textbf{Recall} & \textbf{Parameters} \\
\midrule
CNN & 0.75 & 0.84 & 0.8$\times$10$^6$ \\
RNN & 0.82 & 0.79 & 0.3$\times$10$^6$ \\
LSTM & 0.91 & 0.80 & 0.5$\times$10$^6$ \\
\bottomrule
\end{tabular}
\caption{Esempio di tabella.}
\label{tab:esempio}
\end{table}

\section{Sintassi e lessico}

Di seguito alcune regole da seguire nella stesura della tesi:
\begin{itemize}
 \item evitare l'uso della prima persona plurale o singolare;
 \item evitare l'uso della virgola per unire proposizioni coordinate. Preferire il punto e virgola, una congiunzione o in alternativa riscrivere il periodo;
 \item utilizzate correttamente e opportunamente la terminologia tecnica: ad es., non scrivete \emph{``L'utente fa il login''}, ma \emph{``L'utente effettua l'accesso''}.
\end{itemize}