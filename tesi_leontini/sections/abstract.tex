% Se la tesi è in inglese, mettere prima l'abstract in inglese

\selectlanguage{italian}
\cleardoublepage
\null
\begin{center}
\bfseries
\abstractname
\end{center}
/* bozza */
Nel campo in rapida evoluzione dell'informatica, la capacità di visualizzare e analizzare efficacemente i dati è fondamentale per prendere decisioni informate e sviluppare sistemi efficienti. I sistemi tradizionali di visualizzazione dei dati sono spesso caratterizzati da rigidità e scarsa riusabilità, risultando difficili da adattare a nuovi requisiti o da integrare con tecnologie moderne. Questa tesi affronta tali problematiche proponendo un'architettura di sistema modernizzata e riusabile per la visualizzazione dei dati, sfruttando principi contemporanei dell'ingegneria del software come la modularità, il design basato su componenti e la separazione delle responsabilità. Il progetto analizza i paradigmi e le tecnologie esistenti, ne identifica i limiti e introduce un framework flessibile, capace di presentare dati eterogenei su diverse piattaforme. Attraverso l'implementazione di componenti riusabili e interfacce dati standardizzate, la soluzione proposta facilita una rapida adattabilità ai cambiamenti nelle esigenze degli utenti e nei progressi tecnologici. L'efficacia del sistema viene dimostrata tramite casi di studio pratici e analisi delle performance, evidenziando miglioramenti in termini di manutenibilità, scalabilità ed esperienza utente. L'obiettivo di questo lavoro è offrire una solida base per futuri sviluppi in applicazioni data-centriche, promuovendo sostenibilità e innovazione nella progettazione dei sistemi di visualizzazione dei dati.
\vfill

\selectlanguage{english}
\cleardoublepage
\null
\begin{center}
\bfseries
\abstractname
\end{center}
In the rapidly evolving field of computer science, the ability to efficiently display and analyze data is critical for informed decision-making and effective system development. Traditional data display systems often suffer from rigidity and lack of reusability, making them difficult to adapt to new requirements or integrate with modern technologies. This thesis addresses these challenges by proposing a modernized, reusable system architecture for data display, leveraging contemporary software engineering principles such as modularity, component-based design, and separation of concerns. The project explores existing paradigms and technologies, identifies their limitations, and introduces a flexible framework capable of presenting diverse data types across various platforms. Through the implementation of reusable components and standardized data interfaces, the proposed solution facilitates rapid adaptation to changing user needs and technological advancements. The system’s effectiveness is demonstrated through practical case studies and performance analyses, highlighting improvements in maintainability, scalability, and user experience. Ultimately, this work aims to contribute a robust foundation for future developments in data-centric applications, promoting sustainability and innovation in the design of data display systems.
\vfill
